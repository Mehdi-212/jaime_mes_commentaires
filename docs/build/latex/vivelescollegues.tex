%% Generated by Sphinx.
\def\sphinxdocclass{report}
\documentclass[letterpaper,10pt,french]{sphinxmanual}
\ifdefined\pdfpxdimen
   \let\sphinxpxdimen\pdfpxdimen\else\newdimen\sphinxpxdimen
\fi \sphinxpxdimen=.75bp\relax
\ifdefined\pdfimageresolution
    \pdfimageresolution= \numexpr \dimexpr1in\relax/\sphinxpxdimen\relax
\fi
%% let collapsible pdf bookmarks panel have high depth per default
\PassOptionsToPackage{bookmarksdepth=5}{hyperref}

\PassOptionsToPackage{warn}{textcomp}
\usepackage[utf8]{inputenc}
\ifdefined\DeclareUnicodeCharacter
% support both utf8 and utf8x syntaxes
  \ifdefined\DeclareUnicodeCharacterAsOptional
    \def\sphinxDUC#1{\DeclareUnicodeCharacter{"#1}}
  \else
    \let\sphinxDUC\DeclareUnicodeCharacter
  \fi
  \sphinxDUC{00A0}{\nobreakspace}
  \sphinxDUC{2500}{\sphinxunichar{2500}}
  \sphinxDUC{2502}{\sphinxunichar{2502}}
  \sphinxDUC{2514}{\sphinxunichar{2514}}
  \sphinxDUC{251C}{\sphinxunichar{251C}}
  \sphinxDUC{2572}{\textbackslash}
\fi
\usepackage{cmap}
\usepackage[T1]{fontenc}
\usepackage{amsmath,amssymb,amstext}
\usepackage{babel}



\usepackage{tgtermes}
\usepackage{tgheros}
\renewcommand{\ttdefault}{txtt}



\usepackage[Sonny]{fncychap}
\ChNameVar{\Large\normalfont\sffamily}
\ChTitleVar{\Large\normalfont\sffamily}
\usepackage{sphinx}

\fvset{fontsize=auto}
\usepackage{geometry}


% Include hyperref last.
\usepackage{hyperref}
% Fix anchor placement for figures with captions.
\usepackage{hypcap}% it must be loaded after hyperref.
% Set up styles of URL: it should be placed after hyperref.
\urlstyle{same}

\addto\captionsfrench{\renewcommand{\contentsname}{Contenu:}}

\usepackage{sphinxmessages}
\setcounter{tocdepth}{1}



\title{ViveLesCollegues}
\date{janv. 16, 2023}
\release{0.1}
\author{Mehdi}
\newcommand{\sphinxlogo}{\vbox{}}
\renewcommand{\releasename}{Version}
\makeindex
\begin{document}

\ifdefined\shorthandoff
  \ifnum\catcode`\=\string=\active\shorthandoff{=}\fi
  \ifnum\catcode`\"=\active\shorthandoff{"}\fi
\fi

\pagestyle{empty}
\sphinxmaketitle
\pagestyle{plain}
\sphinxtableofcontents
\pagestyle{normal}
\phantomsection\label{\detokenize{index::doc}}


\sphinxAtStartPar
\sphinxstylestrong{Contexte du projet} :

\sphinxAtStartPar
Vous êtes responsable de la mise à jour de la liste des employés de votre entreprise.
Il s’agit d’une liste enregistrée dans un fichier .txt, et disponible sur un serveur Gitlab distant.
A chaque nouvelle arrivée, vous devez ajouter le nouvel employé dans cette liste.

\begin{sphinxadmonition}{note}{Note:}
\sphinxAtStartPar
Ce projet est en cours de developpement.
\end{sphinxadmonition}


\chapter{Contenu}
\label{\detokenize{index:contenu}}
\sphinxstepscope


\section{Installation du Projet}
\label{\detokenize{Installation:installation-du-projet}}\label{\detokenize{Installation::doc}}\phantomsection\label{\detokenize{Installation:installation}}
\sphinxAtStartPar
Voici le lien du projet, vous pouvez maintenant le télecharger : \sphinxhref{https://gitlab.com/simplonclermontia3/vivelescollegues\_mehdi}{Ici}


\subsection{Tkinter}
\label{\detokenize{Installation:tkinter}}
\sphinxAtStartPar
Pour installer tkinter :

\begin{sphinxVerbatim}[commandchars=\\\{\}]
\PYG{g+gp}{\PYGZdl{} }pip install tk
\end{sphinxVerbatim}


\bigskip\hrule\bigskip



\subsection{tkcalendar}
\label{\detokenize{Installation:tkcalendar}}
\sphinxAtStartPar
Pour installer tkcalendar :

\begin{sphinxVerbatim}[commandchars=\\\{\}]
\PYG{g+gp}{\PYGZdl{} }pip install tkcalendar
\end{sphinxVerbatim}


\bigskip\hrule\bigskip



\subsection{python\sphinxhyphen{}csv}
\label{\detokenize{Installation:python-csv}}
\sphinxAtStartPar
Pour installer python\sphinxhyphen{}csv :

\begin{sphinxVerbatim}[commandchars=\\\{\}]
\PYG{g+gp}{\PYGZdl{} }pip install python\PYGZhy{}csv
\end{sphinxVerbatim}

\sphinxstepscope


\section{Fonction}
\label{\detokenize{Fonction:fonction}}\label{\detokenize{Fonction::doc}}\index{envoyer() (dans le module form\_add)@\spxentry{envoyer()}\spxextra{dans le module form\_add}}

\begin{fulllineitems}
\phantomsection\label{\detokenize{Fonction:form_add.envoyer}}
\pysigstartsignatures
\pysiglinewithargsret{\sphinxcode{\sphinxupquote{form\_add.}}\sphinxbfcode{\sphinxupquote{envoyer}}}{}{}
\pysigstopsignatures
\sphinxAtStartPar
Fonctions permettant d’envoyer la liste des employées

\end{fulllineitems}


\sphinxAtStartPar
L’application ouvre un formulaire dans une fenêtre sur votre ordinateur.
L’utilisateur devra renseigner les informations suivantes : nom, prénom, la date d’entrée et son poste.
Dès que les champs du formulaire seront remplis, il faudra valider via le bouton “Envoyer”.

\noindent{\hspace*{\fill}\sphinxincludegraphics[width=314.50000\sphinxpxdimen,height=335.00000\sphinxpxdimen]{{add_form}.png}}

\sphinxstepscope


\section{Utilisation de Git}
\label{\detokenize{Utilisation:utilisation-de-git}}\label{\detokenize{Utilisation::doc}}
\sphinxAtStartPar
Voici les étapes pour un bon fonctionnement de projet avec Git.

\sphinxAtStartPar
Nous avons commencé par créer des tickets sur Gitlab:

\sphinxAtStartPar
\sphinxstylestrong{Creation du formulaire}
\begin{quote}

\sphinxAtStartPar
\sphinxstyleemphasis{Début du formulaire fait !}
\end{quote}


\bigskip\hrule\bigskip


\sphinxAtStartPar
\sphinxstylestrong{Création des fonctions necessaire au bon fonctionnement du formulaire}
\begin{quote}

\sphinxAtStartPar
\sphinxstyleemphasis{Création des fonctions necessaire au bon fonctionnement du formulaire}
\begin{itemize}
\item {} 
\sphinxAtStartPar
Liste déroulante/Fichier TXT

\item {} 
\sphinxAtStartPar
Fonction envoyer pour le formulaire + fichier csv

\end{itemize}
\end{quote}


\bigskip\hrule\bigskip


\sphinxAtStartPar
\sphinxstylestrong{Faire l’interface graphique}
\begin{quote}

\sphinxAtStartPar
\sphinxstyleemphasis{Création du formulaire avec 4 champs (nom,prenom, date, profession) + un bouton envoyer}
\end{quote}


\bigskip\hrule\bigskip


\sphinxAtStartPar
\sphinxstylestrong{Convention}
\begin{quote}

\sphinxAtStartPar
\sphinxstyleemphasis{Le premier caractère de Nom et Prénom est mis en majuscule.}
\end{quote}


\bigskip\hrule\bigskip


\sphinxAtStartPar
\sphinxstylestrong{Liste déroulante/Fichier TXT}
\begin{quote}

\sphinxAtStartPar
\sphinxstyleemphasis{Création de la liste déroulante pour les professions.
Création du fichier TXT pour les professions.
Ajout des professions dans le fichier TXT.}
\end{quote}


\bigskip\hrule\bigskip


\sphinxAtStartPar
\sphinxstylestrong{Calendrier}
\begin{quote}

\sphinxAtStartPar
\sphinxstyleemphasis{Modifier le champ date par un calendrier}
\end{quote}


\bigskip\hrule\bigskip


\sphinxAtStartPar
\sphinxstylestrong{Fonction envoyer pour le formulaire + fichier csv}
\begin{quote}

\sphinxAtStartPar
\sphinxstyleemphasis{Ajout du fichier csv et creation de la fonction qui permet d’envoyer les informations dans le fichier csv}
\end{quote}


\bigskip\hrule\bigskip


\sphinxAtStartPar
Voici les commandes les plus utilisées:

\noindent{\hspace*{\fill}\sphinxincludegraphics[width=1.000\linewidth]{{Post_2_GIT_Life_Cycle}.png}\hspace*{\fill}}


\bigskip\hrule\bigskip


\sphinxAtStartPar
Nous avons utilisé le logiciel \sphinxstylestrong{Fork} pour effectuer ces tâches.

\noindent{\hspace*{\fill}\sphinxincludegraphics[width=1.000\linewidth]{{merge_sophana_branch}.jpg}\hspace*{\fill}}


\bigskip\hrule\bigskip


\sphinxstepscope


\section{Données utilisés pour le formulaire}
\label{\detokenize{donnees:donnees-utilises-pour-le-formulaire}}\label{\detokenize{donnees::doc}}

\subsection{Fichier en csv}
\label{\detokenize{donnees:fichier-en-csv}}
\sphinxAtStartPar
Tableau avec tous les employés enregistrés à partir du formulaire


\begin{savenotes}\sphinxattablestart
\centering
\begin{tabulary}{\linewidth}[t]{|T|T|T|T|}
\hline

\sphinxAtStartPar
Nom
&
\sphinxAtStartPar
Prénom
&
\sphinxAtStartPar
Date d’embauche
&
\sphinxAtStartPar
Poste
\\
\hline
\sphinxAtStartPar
Roberts
&
\sphinxAtStartPar
Julia
&
\sphinxAtStartPar
10/10/2022
&
\sphinxAtStartPar
Patronne
\\
\hline
\sphinxAtStartPar
Doe
&
\sphinxAtStartPar
John
&
\sphinxAtStartPar
11/10/2022
&
\sphinxAtStartPar
Ingénieur
\\
\hline
\sphinxAtStartPar
Depp
&
\sphinxAtStartPar
Johnny
&
\sphinxAtStartPar
11/10/2022
&
\sphinxAtStartPar
Concierge
\\
\hline
\sphinxAtStartPar
Parker
&
\sphinxAtStartPar
Peter
&
\sphinxAtStartPar
01/10/2022
&
\sphinxAtStartPar
Stagiaire
\\
\hline
\end{tabulary}
\par
\sphinxattableend\end{savenotes}

\sphinxstepscope


\section{Tâches effectuées en équipe}
\label{\detokenize{tache:taches-effectuees-en-equipe}}\label{\detokenize{tache::doc}}
\sphinxAtStartPar
Chacun de nous avions des tâches différentes et créé une branche pour les modifications.

\sphinxAtStartPar
\sphinxstylestrong{Légende:}
\begin{itemize}
\item {} 
\sphinxAtStartPar
Mehdi (M)

\item {} 
\sphinxAtStartPar
Sophana (S)

\end{itemize}

\sphinxAtStartPar
\sphinxstylestrong{Liste des tâches par ordre:}
\begin{itemize}
\item {} 
\sphinxAtStartPar
Réalisation d’un formulaire avec 4 champs et un bouton. Ouverture avec Tkinter (S)

\item {} 
\sphinxAtStartPar
Création d’une fonction pour enregistrer les valeurs du champs dans un fichier CSV (M)

\item {} 
\sphinxAtStartPar
Modification du champ « Date » par un widget calendrier (S)

\item {} 
\sphinxAtStartPar
Modification du champ « Profession » par une liste déroulante où celle\sphinxhyphen{}ci se met à jour via un fichier txt (M)

\item {} 
\sphinxAtStartPar
Mise à jour de fonction permettant d’enregistrer dans le CSV. Le nom et le prénom commenceront toujours par une lettre majuscule (S)

\end{itemize}

\sphinxstepscope


\section{Rapport final}
\label{\detokenize{Rapport:rapport-final}}\label{\detokenize{Rapport::doc}}
\sphinxAtStartPar
Un rapport explique les choix effectué pour l’automatisation du fichier et le découpage des tâches nécessaires au sein du binôme.


\bigskip\hrule\bigskip

\begin{itemize}
\item {} 
\sphinxAtStartPar


\item {} 
\sphinxAtStartPar


\item {} 
\sphinxAtStartPar


\end{itemize}


\bigskip\hrule\bigskip



\chapter{Indices and tables}
\label{\detokenize{index:indices-and-tables}}\begin{itemize}
\item {} 
\sphinxAtStartPar
\DUrole{xref,std,std-ref}{genindex}

\item {} 
\sphinxAtStartPar
\DUrole{xref,std,std-ref}{modindex}

\item {} 
\sphinxAtStartPar
\DUrole{xref,std,std-ref}{search}

\end{itemize}



\renewcommand{\indexname}{Index}
\printindex
\end{document}